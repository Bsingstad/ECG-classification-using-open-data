Cardiovascular diseases is one of the leading causes of death in the world. Numbers from WHO estimates that $17.9$ million people died from cardiovascular death (CVD) in 2016 which represented $31\%$ of all global death that year \cite{noauthor_cardiovascular_nodate}. Early detection of patient with risk of CVD could potentially decrease the amount of CVD. Electrocardiography is a method with a huge potential for detecting patients with risk of CVD. The electrocardiograph is non-invasive and relativly easy to use, compared to methods like echocardiogram and MRI, which makes it an convenient diagnostic tool. As an example of how widely the electrcardigraph is used, National Ambulatory Medical Care reported that 40 millions electrocardiograms (ECG) were recorded in USA in 2015 \cite{us_department_of_health_and_human_services_national_2015}.

An electrocardiograph measures the electrical activity of the heart from electrodes placed on the surface of the upper body. The result of such a measurement is an ECG. The ECG is a graphical representation of the measured electrical activity of the heart with respect to time. One of the challenges is that the ECG can be difficult to interpret correctly. The interpretation can be time consuming and require a high degree of expertise \cite{bickerton_misplaced_2019}.

Many of the modern and clinically used electrocardiographs today are equipped with a built-in interpretation program. The interpretation program analyzes the ECG and prints interpretive texts that may indicate different diseases. Studies show that there are some limitations to the automatic interpretation algorithms \cite{schlapfer_computer-interpreted_2017, smulyan_computerized_2019}. The errors, caused by the automatic interpretation algorithms, means that doctors or cardiologists has to read over the ECGs to ensure they are correct.

The hypothesis in this study is that machine learning can improve today's interpretive algorithms. Eight machine learning models from a previous study \cite{singstad_convolutional_nodate} will be evaluated and compared with two new machine learning models, were one of them will utilize 12 lead and the other will utilize only 2 leads. 

A considerable amount of literature has been published on heartbeat classification \cite{annam_classification_2020}, single \cite{mathews_novel_2018} and even  2-lead classifiaction \cite{liu_arrhythmia_2013} over the last ten years . In most recent years there have been an increasing focus on 12-lead ECG classificaion and some recent studies has shown that machine learning is feasible \cite{ribeiro_automatic_2020, yao_multi-class_2020,li_automatic_2020, chen_detection_2020}. On the other hand, the  dataset used has either been small and homogeneous \cite{noauthor_classification_nodate} or not accessible to everyone. In this study, a large, open dataset from several sources and a large variation in different diagnoses will be examined and used as development set for training machine learning models \cite{alday_classification_2020}. This dataset was used in a challenged held by PhysioNet \cite{goldberger_physiobank_2000} and Computing in Cardiology (CinC) in 2020 were $217$ teams submitted $1395$ algorithms during the challenge \cite{alday_classification_2020}. A training set and a test set were provided and the team who got the best score on the test set won the competition. The best team called themselves \textit{prna} and they achieved a PhysioNet/CinC Challenge score \cite{alday_classification_2020} of $0.533$ on the test set and a cross-validated score of $0.533\pm 0.046$ (mean and standard deviation). 

It is already stated that PhysioNet/CinC Challenge 2021 will utilize the same dataset, but this time investigate both 12-lead and 2-lead ECG in a challenge called "will 2 to?". One of the objective of this study is to prepare an initial submission to the PhysioNet/CinC Challenge 2021 which will go live in the end of December 2020.

In addition, this study will demonstrate how to get an explainable prediction from the machine learning models developed in this study. Explainability of machine learning models is a new field in artificial intelligence (AI) and is called explainable AI. Explainable predictions are very important in medical diagnostics. As an example, cardiologists and doctors can use their knowledge to see if the parameters used for the prediction, by an ECG classification model, can be explained physiologically. This will probably lead to better trustworthiness for an ECG classification algorithm among cardiologists and doctors.